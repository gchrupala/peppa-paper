\appendix

\section{Supplementary material}



\subsection{Targeted Triplets Evaluation Sets}\label{app:targeted_triplets_eval}

To find commonly occurring nouns, adjectives, and verbs, we lemmatize and POS-tag all words in the transcripts of the validation dataset using spacy \citep{honnibal2020spacy}. Afterwards, we identify sets of all nouns $\{n_1, ..., n_n\}$, verbs $\{v_1, ..., v_o\}$ and adjectives $\{a_1, ..., a_p\}$ that occur at least 10 times in the validation data. Given these sets, we create sets of tuples $\{(n_1, n_2), (n_1, n_3), ..., (n_1, n_n), ...,  (n_{n-1}, n_n)\}$ for all combinations of nouns and verbs, respectively. For each of these tuples, we search the validation data for pairs of phrases $(p_k=[w_1, ..., w_x], p_l=[w_1, ..., w_y])$ with same length ($x=y$) and minimal difference regarding the tuple. That is, $n_1 \in p_1$, $n_2 \in p_2$, and if we replace $n_1$ with $n_2$ in $p_1$, it is equal to $p_2$. 

For example, if $n_1 = \text{"peppa"}$ and $n_2 = \text{"george"}$, the phrases $p_1 = [\text{"peppa", "loves", "jumping"}]$ and $p_2 = [\text{"george", "loves", "jumping"}]$ are phrases with minimal differences. A phrase can also be a single word.

We set the minimum phrase duration to 0.3 seconds (for shorter sequences, we do not expect that the video data contains enough semantic information for a model to distinguish between target and distractor). For each phrase $p_1$ we look for the \textit{longest} possible phrase $p_2$. \Cref{fig:num_samples_vs_duration} shows the distribution of samples per duration, \Cref{fig:num_samples_vs_num_tokens} per number of tokens.

Based on each minimal pair, we construct two counter-balanced test triplets as described in the main text.

Figures \ref{fig:num_samples_ADJ_word}, \ref{fig:num_samples_NOUN_word}, and \ref{fig:num_samples_VERB_word} show the number of samples for each adjective, noun and verb.


\begin{figure}
  \centering
  \includegraphics[width=\textwidth]{results/targeted_triplets/num_samples_vs_duration.png}
  \caption{Number of samples per duration}
  \label{fig:num_samples_vs_duration}
\end{figure}


\begin{figure}
  \centering
  \includegraphics[width=\textwidth]{results/targeted_triplets/num_samples_vs_num_tokens.png}
  \caption{Number of samples per number of tokens}
  \label{fig:num_samples_vs_num_tokens}
\end{figure}


\begin{figure}
  \centering
  \includegraphics[width=\textwidth]{results/targeted_triplets/num_samples_ADJ_word.png}
  \caption{Number of samples: adjectives}
  \label{fig:num_samples_ADJ_word}
\end{figure}

\begin{figure}
  \centering
  \includegraphics[width=\textwidth]{results/targeted_triplets/num_samples_NOUN_word.png}
  \caption{Number of samples: nouns}
  \label{fig:num_samples_NOUN_word}
\end{figure}

\begin{figure}
  \centering
  \includegraphics[width=\textwidth]{results/targeted_triplets/num_samples_VERB_word.png}
  \caption{Number of samples: verbs}
  \label{fig:num_samples_VERB_word}
\end{figure}

