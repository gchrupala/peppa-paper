\documentclass[a4paper]{article}
\usepackage{etoolbox}
\usepackage{url}
\usepackage[T1]{fontenc}
\usepackage[hidelinks=true]{hyperref}
\usepackage{appendix}
\usepackage{natbib}
\usepackage[utf8x]{inputenc}
\usepackage{booktabs}
%\setlength {\marginparwidth }{2cm}
\usepackage{todonotes}
\usepackage{adjustbox}
\usepackage{microtype}
\usepackage{multirow}
\usepackage{placeins}
\usepackage{longtable}
\usepackage{enumitem}
\usepackage{amsmath}
\usepackage{cleveref}
\usepackage{wrapfig}
\DeclareMathOperator*{\mean}{mean}
%%%%%%%%%%%%%%%%%%%%%%%%%%%%%%%%%%%%%%%%%%%%%%%%%%%%%%%%

\begin{document}

\title{Learning English with Peppa Pig}

\author{}
\date{}


\maketitle
\begin{abstract}
  Attempts to computationally model or simulate the acquisition of
  spoken language via grounding in the visual modality have a long
  tradition but have gained momentum since around 2015 with the
  revival of neural networks. Current neural approaches are able to
  spot associations between the spoken and visual modality, and use
  these to represent speech and image/video data in a joint vector
  space. A major limitation of these works are the datasets used to
  train them. Most consist of static images or videos paired with
  spoken descriptions of what is depicted, and thus guarantee a strong
  correlation between speech and the visual world by construction. A
  child learning a language faces a very different and harder task: in
  the real world the coupling between the linguistic and the visual is
  much looser, and often contains confounds in the form of
  correlations with non-semantic aspects of the speech signal, such as
  voices of specific people and environmental sounds. The current
  study is a first step towards simulating such a naturalistic
  grounding scenario by using a dataset based on the children's
  cartoon {\it Peppa Pig}. We train a simple bi-modal architecture on
  the portion of the data consisting of naturalistic dialog between
  characters, and evaluate on segments containing descriptive
  narrations. Evaluation and analysis results indicate that despite
  the weak and confounded signal in this training data our model
  succeeds at learning aspects of the visual semantics of spoken
  language.
\end{abstract}

\section{Introduction}
\label{sec:intro}

Attempts to model or simulate the acquisition of spoken language via
grounding in the visual modality date to the beginning of this century
\citep{roypentland2002learning} but have gained momentum recently
with the revival of neural networks
\citep[e.g.][]{synnaeve2014learning,harwath2015deep,
  harwath2016unsupervised,chrupala-etal-2017-representations,alishahi-etal-2017-encoding,harwath2018jointly,Merkx2019,havard2019models,rouditchenko2020avlnet,khorrami_2021,peng2021fastslow}.
Current approaches work well enough from an applied point of view, 
but most are not generalizable to real-life situations that humans or 
adaptive artificial agents experience. Commonly used training data
consist of images or videos paired with spoken descriptions
of the scene depicted: however, the type of input that a language learner receives 
from the environment is much more challenging.  
Firstly, speech is only loosely coupled with the visual modality in naturalistic settings
 \citep{matusevych2013automatic, beekhuizen2013word}. Speakers often mention 
 concepts that are not present in the immediate perceptual context, 
 or talk about events that are remote in space and/or time (for
 example past experiences or future plans).
 
Secondly, in addition to
correlations between the visual scenes and the {\it meaning} of spoken
utterances, there are also correlations with non-semantic aspects of
the speech signal, such as the voices of specific speakers, as well
as with non-speech ambient sounds. Although it is plausible that such
non-semantic correlations can sometimes be useful to the learner in
the general endeavor of making sense of the world, for the specific
task of learning the semantics of linguistic units they are likely more
often an obstacle, as they make it harder to zoom in on the
meaning-bearing aspects of the audio signal.

In the current study we make a first step towards simulating the
acquisition of language via grounding in perception in a more
naturalistic scenario.  Our main focus is on learning the meaning of
linguistic expressions from spoken utterances grounded in video.  We
use the well-known children's cartoon {\it Peppa Pig} as a case
study. Compared to commonly used video datasets, this dataset has a
number of interesting characteristics.  The visual modality is very
schematic, the language is simple in terms of vocabulary size and
syntactic complexity, and analysis of its linguistic features suggests
its suitability for beginner learners of English
\cite{kokla2021peppa,scheffler2021peppa}.  Crucially, however, most of
the speech in the videos consists of naturalistic dialogs between the
characters in which they do not only discuss the here and now, but
also often use displaced language.\footnote{For example, when Daddy
  Pig explains that they need to clean up before Mummy Pig sees the mess
  Peppa and George made, or when talking about plans to visit
  friends.}  Thus, the utterances are only loosely and noisily
correlated to the scenes and actions depicted in the videos.

This choice of data thus allows us to directly address the ecological limitations 
of the current approaches. In addition, the cartoon videos also contain 
comments interjected by the narrator. We use these for evaluating the 
acquisition of meaning as they are more descriptive and less noisy and allow 
us to measure performance, while controlling for speaker characteristics.

We implement a simple bi-modal architecture which learns spoken
language embeddings from videos, and train it on the Peppa Pig dataset.
Our contributions are the following:
\begin{itemize}
\item We evaluate model performance in terms of video fragment
  retrieval and additionally design controlled evaluation
  protocols inspired by the intermodal preferential looking
  paradigm \citep{hirsh1996intermodal};
\item We carry out ablations of model components in order to
  understand the effects of pre-training for the audio and video
  encoders, the role of temporal information, and of segmentation
  strategies while training. 
\end{itemize}
We show that despite the challenges of our naturalistic training data,
our model succeeds at learning associations between the form of spoken 
utterances and their visual semantics. Moreover, even though the model 
rarely hears words in isolation, it captures aspects of the visual meaning 
of frequent nouns and verbs.
Our ablation studies suggest that temporal information contributes
to video modeling (especially for longer segments), and that self-supervised pre-training
followed by fine-tuning of the audio encoder is key to the best
performance.




\section{Related Work}
\label{sec:related}

Early attempts at simulating grounded language learning focus on
interactions between adults and young children while playing with a
set of objects from different categories \cite{roy1999learning,
  roy2000grounded, roy2000learning, roy2002learning,
  gorniak2003visually, mukherjee2003visual}. In a representative study
from these series, \citet{roypentland2002learning} use speech recorded from
such interactions paired with different views of the visible objects
to identify linguistic units (i.e.\ words) and visual categories, and
to map these two modalities together. A hard-coded visual system
extracts object representations from images, and spoken utterances are
represented as phoneme probabilities generated by a RNN pre-trained on
spectrograms.  Their experiments on small-scale data (around 20 words
and seven visual categories) show that the model can segment words and
map them to visual categories.

\subsection{Spoken Language Grounded in Images}
\label{sec:images}
The availability of datasets of images associated with spoken captions
such as Flickr Audio Captions \citep{harwath2015deep}, Places
\cite{zhou2014learning} and Spoken COCO \citep{hsu2019transfer} led to
a rapid development of deep models of grounded language learning; see
\citet{chrupala-visually-2021} for a comprehensive overview. \todo{MN: maybe 
highlight here the difference of these models to the early approaches by deb 
roy (e.g. handling larger + more realistic data, ..)?}
 Most of
these models adapt the architecture of \citet{karpathy2014deep} by
implementing separate pathways for encoding the visual and speech
modality, and mapping these encodings into a joint representation
space. Visual features are extracted from a pre-trained
image classification model that processes the whole or a specific
region of an image (however see \citet{harwath2018jointly}, who train the
model end-to-end on images and their spoken captions on the Places
dataset). The audio encoder component in most models is 
either an adaptation of \citet{harwath2016unsupervised} which feeds a
spectrogram of the speech signal to a convolutional architecture, or a
hybrid architecture of convolutional followed by recurrent layers using
Mel-Frequency Cepstral Coefficient (MFCC) features from the audio
signal as input as introduced by \citet{chrupala-etal-2017-representations}.

Models of speech grounded in images are optimized for and evaluated on
image retrieval from spoken caption and vice versa. Additionally, a range of
diagnostic analyses have been performed on the hidden
representations of these models to study whether they encode knowledge
about the identity and boundaries of subword units such as phonemes
and syllables \cite{alishahi-etal-2017-encoding, harwath2019towards,
  khorrami_2021} as well as individual words
\cite{chrupala-etal-2017-representations,havard2019word}. Moreover, in
addition to examining form-meaning associations at the utterance
level, \citet{harwath2017learning} explicitly learn a lexicon by
extracting audio and image segments, clustering each modality
separately, and mapping them together by calculating the pairwise
similarities of their members in the joint semantic space.

\subsection{Spoken Language Grounded in Video}
\label{sec:video}
There have been also been recent attempts to learn spoken language grounded
in video instead of static images.  \citet{boggust2019grounding}
sample audio-visual fragments from cooking videos, however their
grounded model treats video frames as still images and discard their
temporal order.
% The loose synchrony between the two modalities, such that objects
% may be mentioned in the audio at a different point in time than they
% occur in the video, remains the main challenge for this approach.
\citet{rouditchenko2020avlnet} integrate the temporal information when
encoding videos from the Howto100m dataset \cite{miech2019howto100m},
and perform better than previous work in language and video clip
retrieval.
% Rouditchenko et al. (2021) present an architecture (AVLNet) which
% does model the time dimension in the video stream: the network
% consists of an audio encoder (ResNet-based), a video encoder which
% combines 3D and 2D modeling (also ResNet-based), as well as an
% optional text encoder.  This architecture is trained with a
% contrastive loss on randomly sampled audio-video fragments from the
% Howto100m dataset (Miech et al., 2019) consisting of 136 million
% video clips sourced from 1.22 million narrated instructional web
% videos. The model is evaluated on the video clip and language
% retrieval tasks on smaller video datasets annotated with clip
% boundaries and text summaries, and is shown to outperform previously
% proposed models of Arandjelovic and Zisserman (2018) and Boggust et
% al. (2019). The model also transfers to the image-audio retrieval
% setting. Qualitative analysis suggests that the model aligns
% semantically related audio and visual features to particular
% dimensions of the embedding space
Models trained on such instructional video datasets often do not
generalize well to other domains. \citet{monfort2021spokenmoments}
highlight this limitation and show that training on their larger and
more diverse Spoken Moments in Time dataset leads to better
generalization.  The point remains that these video datasets contain
descriptive speech, thus ensuring that there is a strong correlation
between the spoken language and their visual context, a characteristic
that is not representative of the experience of learning language in
the real world.

\paragraph{Child language learning from video.}
% \todo[inline]{I'm still cleaning up the section on infant language
% learning from video.}  In addition to anecdotes and news pieces on
% how watching cartoons affects chidlren's accent and
% vocabulary\footnote{Example of stories on how watching Peppa Pig has
% changed children's accent and word usage:
% \url{https://www.theguardian.com/tv-and-radio/2021/jul/19/peppa-pig-american-kids-british-accents},
% \url{https://globalnews.ca/news/4961058/kids-accent-peppa-pig}.},
There are many studies on young children learning language by watching
videos; see \citet{vanderplank2010deja} for a survey. The main takeaway
of these studies is that language learning is much more effective in a
social, conversational setting than by passively watching videos
\cite{kuhl2003foreign,anderson2005television,robb2009just},\footnote{Adding
social interaction while watching videos improves learning;
\citet{lytle2018two}.} but learning does happen in such
contexts. Importantly for our goal, techniques such as the intermodal
preferential looking paradigm have been developed to systematically test young 
language learners' knowledge of words, syntactic structure and semantic roles
\cite{hirsh1996intermodal,bergelson20126,noble2011comprehension}.
\citet{nikolaus-fourtassi-2021-evaluating}
employ this evaluation strategy to test semantic knowledge at word and
sentence level in their computational model of word learning from
images. We adapt this approach to evaluate how our grounded model
associates semantic information to spoken words and utterances from
video.
%The paradigm has been used in language acquisition research to
%evaluate children's early linguistic knowledge
%\citep[e.g.,][]{noble2011comprehension,bergelson20126}, by testing
%whether they can distinguish a matching (target) visual referent from
%a foil (distractor) referent when prompted with a word or
%sentence.
 
 
% \citep[e.g.][]{synnaeve2014learning,harwath2015deep,
% harwath2016unsupervised,chrupala-etal-2017-representations,alishahi-etal-2017-encoding,harwath2018jointly,Merkx2019,havard2019models,rouditchenko2020avlnet,khorrami_2021,peng2021fastslow}.



% \paragraph{Audiovisual models}
% \citet{aytar2016soundnet,owens2016visually,owens2016ambient}
% \paragraph{Video captioning}
% \citet{krishna2017dense,zhou2018end}
\todo{GC: I think we need a paragraph or so on recent work on
  self-supervised learning from multimodal data. SOme possible works
  to mention:
  https://arxiv.org/abs/2104.11178, https://arxiv.org/abs/2103.03206}

\section{Method}
\label{sec:method}

\subsection{Dataset}
The dataset consists of the complete set of videos of the
English-language version of {\it Peppa Pig}. In addition to the raw
videos we  also use the annotation created by
\citep{papasarantopoulos2021narration}.

These annotations feature written transcriptions of the audio as well
as segmentation into {\it dialog} and {\it narration}. Dialogs are the
parts spoken by the characters, while narrations are comments inserted
by the narrator, which are more descriptive in nature. All the narration
segments are uttered by the same actor. We use the dialogs for
training the model, and set aside the narrations for evaluation
purposes only.

Specifically, we use dialog from episodes 1--196 for training,
197--202 for validation and 203-209 for testing. We set aside
narrations from episodes 1--104 for validation and 105--209 for
testing. 


\subsection{Preprocessing}
For training, we do not use word or sentence level segmentation in
order to make the setting more naturalistic. Instead we split the
dialog sections into 3.2 second non-overlapping fragments. The video
is subsampled to 10 frames per second, and to 180x100 resolution. The
audio is converted to mono by averaging the two channels  and the raw
waveform is used as input.


For evaluation we have a number of different conditions and evaluation
metrics described in detail in \Cref{sec:eval} and in some of these
conditions we use the subtitles to guide
segmentation. \Cref{tab:ds-stat} shows the basic statistics of the
training and validation splits.
\todo{Add the test split.}

\begin{table}
  \centering
  \begin{tabular}{lllrrr}
    \toprule
    Split      & Type      & Triplet   & Size (h) & Items & Mean
                                                            length
                                                            (s)\\\midrule
    Training   & Dialog    & No        & 9.83     & 11,058 & 3.2 \\
    Validation & Dialog    & No        & 0.33     & 375    & 3.2 \\
    Validation & Narration & No        & 0.80     & 897    & 3.2 \\
    Validation & Dialog    & Yes       & 0.16     & 202    & 2.8 \\
    Validation & Narration & Yes       & 0.45     & 726    & 2.2 \\
    \bottomrule
  \end{tabular}
  \caption{Dataset statistics. For the triplet condition, videos are
    split such that each segment corresponds to a line of
    subtitles. For the non-triplet condition, videos are split into
    3.2s segments.}
  \label{tab:ds-stat}
\end{table}


\subsection{Evaluation}
\label{sec:eval}
\section{Results}
\label{sec:results}
\paragraph{Performance metrics}
\Cref{tab:scores-dialog} and \Cref{tab:scores-narration} show
the performance of several model configurations on the retrieval and
triplet tasks on the dialog and narration datasets respectively.

In the case of the narration data this scores is not confounded by
speaker-based clues, which is a indication that the model possibly
learned to detect some aspects of utterance meaning. We investigate
this hypothesis further using multiple representational similarity
analysis.
 

 \begin{table}
   \centering
   \input{results/scores_dialog.tex}
   \caption{Retrieval and triplet scores on dialog validation data.}
   \label{tab:scores-dialog}
 \end{table}

\begin{table}
   \centering
   \input{results/scores_narration.tex}
   \caption{Retrieval and triplet scores on narration validation data.}
   \label{tab:scores-narration}
 \end{table}
 
 
\paragraph{Multiple representational similarity analysis}

\begin{figure}
  \centering
  \includegraphics[scale=0.66]{results/grsa_dialog_coef.pdf}
  \caption{Association of predictors with trained and untrained
    model-based pairwise similarity scores for single-word utterances
    in the dialog validation data. Indicators are sum-coded ($1$ vs
    $-1$) while the numerical variables are z-scored.}
  \label{fig:coef_dialog}
\end{figure}

\begin{figure}
  \centering
  \includegraphics[scale=0.66]{results/grsa_narration_coef.pdf}
  \caption{Association of predictors with trained and untrained
    model-based pairwise similarity scores for single-word utterances
    in the narration validation data. Indicators are sum-coded ($1$ vs
    $-1$) while the numerical variables are z-scored.}
  \label{fig:coef_narration}
\end{figure}



The effect of
the {\tt samespeaker} predictor for the dialog data is negative and
small in size.  These results further indicate that the model learns
some aspects of word-level semantics as captured by GloVe word
vectors, and that speaker identity does not appear to be a substantial
impact on utterance embeddings.

Perhaps unexpectedly, the predictor meant to capture phonemic distance
{\tt distance} is not strongly associated with utterance similarity,
although it should be noted that here we are only investigating model
embeddings after the final attention pooling layer.The strength of
the association between differences in utterance duration {\tt
  durationdiff} and pairwise similarities apparent in this data was
suprising and possibly undesirable, and thus warrants further investigation.


\section{Conclusion}
\label{sec:conclusion}
In the real world
the coupling between the linguistic and the visual modality is
loose, and often confounded by correlations with non-semantic
aspects of the speech signal. Here we take a first step in modeling language 
acquisition under such challenging circumstances by
using a dataset based on the children's cartoon {\it Peppa Pig}.  We
train a simple bi-modal architecture on the portion of the data
consisting of dialog between characters, and evaluate on segments
containing descriptive narrations. Despite the weak and confounded
signal in this training data our model succeeds at learning aspects
of the visual semantics of spoken language.
  
We simulate grounded language learning in a naturalistic setting, where 
connection between the linguistic and visual modality is not always strong 
and is potentially confounded by correlations with non-semantic aspects of 
the speech signal. Our experimental results suggest that despite the 
challenges inherent to the naturalistic aspects of our training dataset, a 
simple bimodal architecture can capture aspects of visual meaning of individual 
words as well as full utterances, and generalize well to narrative utterances
featuring a single unseen speaker and a descriptive rather than
conversational style. Our analyses show that generalization is substantially
boosted by fine-tuning audio representations pretrained on unlabeled
single-modality speech data. Fine-tuning a pretrained video encoder
also makes a contribution, but is less crucial to generalization from
dialog to narration.

We also investigate the role of temporal information in learning form-meaning 
mappings. Our various experimental conditions show that having access to 
temporal information facilitates learning, except for very short video segments.

\subsection{Limitations and Future Work}
\label{sec:limitations}

In order to investigate what aspects of spoken language our model
acquires, we would like to carry out in-depth analyses of learned
representations on sub-word, lexical, and phrasal levels. It would
also be worthwhile to figure out the details of how specifically
temporal information in video contributes to acquiring linguistic
knowledge.  Some analyses in this direction are currently constrained
by the size of the evaluation dataset, and more large-scale datasets
are needed in the future.

We model the acquisition of spoken language from 
language-internal correlations as well as from grounding in vision 
by fine-tuning an audio encoder pretrained on read speech. This 
approach is rather simplistic and does not match the real experience of 
language learners. It would be interesting to make the setting
more realistic by using pretraining data which reflect a young
learner's experience more closely, and to realistically interleave learning via
self-supervision from speech and via grounding in vision.
Ideally we would also want to dispense with supervised pretraining of 
the video encoder and rather use a model pretrained in a
self-supervised way also for this modality.

Using naturalistic datasets such as the \textit{Peppa Pig} cartoons,
we believe that it will be possible relate findings more closely to
theories in language acquisition in the real world.

\bibliography{biblio,anthology}
\bibliographystyle{apalike}
\appendix

\section{Supplementary material}

\begin{table}
  \centering
  \begin{adjustbox}{angle=90}
    \input{results/rsa_dialog_correlations.tex}
  \end{adjustbox}
  \caption{Variable correlations, dialog pairwise similarity data.}
  \label{tab:dialogvarcor}
\end{table}
\begin{table}
  \centering
  \begin{adjustbox}{angle=90}
    \input{results/rsa_narration_correlations.tex}
  \end{adjustbox}
  \caption{Variable correlations, narration pairwise similarity data.}
  \label{tab:narrationvarcor}
\end{table}


\end{document}
