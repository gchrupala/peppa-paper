\appendix
\section{Modeling video and speech}


\subsection{Separate encoders}
This is approach taken in \citet{rouditchenko2020avlnet} where the
video encoder and the audio encoder both separately embed their respective
modality as a vector. The loss function is contrastive, based on
max-margin softmax \citep{ilharco-etal-2019-large}, where the softmax
is applied to pairwise dot products between modality-specific vectors:

\begin{equation}
  \label{eq:mms}
  L(\mathbf{x}, \mathbf{y}) = - \mean_i \left(
    \log \frac{\exp(\mathbf{x}_i\cdot \mathbf{y}_i-\delta)}
    { \exp(\mathbf{x}_i\cdot \mathbf{y}_i-\delta) + \sum_{j\neq i}
      \exp(\mathbf{x}_i \cdot \mathbf{y}_j)  }
  \right)
\end{equation}
where $\mathbf{x, y}$ are modality-specific vectors and $delta$ is the
margin. The second term in the denominator sums over the negative
examples.

The advantage of this type of approach is its simplicity and the fact
that there is a single vector representation for the whole speech
fragment.

\subsection{Matchmap}
The matchmap is a concept introduced in \citet{harwath2018jointly} and
which specifies the affinity between each region of the two
dimensional visual feature map and each frame in the associated audio,
i.e.\ it is a three-dimensional tensor. The similarity between the
image and the speech can be computed by a number of scoring functions
which aggregate across these dimensions in different ways: the one
which tended to work best was defined as:
\begin{equation}
  \label{eq:misa}
  \mathrm{MISA}(M) = \mean_t \left(\max_{r,c}(M_{r,c,t})\right)
\end{equation}

The extension of this idea to videos associated with speech would
simply add an extra dimension to the matchmap, corresponding to time
in the visual domain. This would make the matchmap a four-dimensional
tensor, specifying the affinity between each region of the each frame
of the video with a frame of the audio. Overall similarity between
video and speech would involve a scoring function aggregating over
the four dimension: the equivalent to MISA would be:
\begin{equation}
  \label{eq:misa4}
  \mathrm{MISA}_4(M) = \mean_t \left(\max_{r,c,\tau}(M_{r,c,\tau,t})\right),
\end{equation}
where $t$ is time in the audio modality, and $\tau$ is time in the
video modality. This function has the effect that for each frame of
the audio, if finds the best matching frame/region (let us call this
{\it fragment}) of the video, and averages over the scores of these
matches. It does this without encouraging any sort if structure to
this alignment, for example nothing prevents the same fragment of the
video to be the best match for each frame of the audio.

This approach does not per se produce a single vector representing an
utterance and thus if audio-audio retrieval or similarity metric is
desired, something more complex than a simple cosine similarity would
be needed.

\subsection{Dual modality encoder}


\subsection{Audiovisual models}
\citet{aytar2016soundnet,owens2016visually,owens2016ambient}
\subsection{Video captioning}
\citet{krishna2017dense,zhou2018end}