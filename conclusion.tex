\section{Conclusion}
\label{sec:conclusion}
In the real world
  the coupling between the linguistic and the visual modality is
  loose, and often confounded by correlations with non-semantic
  aspects of the speech signal. Here we address this shortcoming by
  using a dataset based on the children's cartoon {\it Peppa Pig}.  We
  train a simple bi-modal architecture on the portion of the data
  consisting of dialog between characters, and evaluate on segments
  containing descriptive narrations. Despite the weak and confounded
  signal in this training data our model succeeds at learning aspects
  of the visual semantics of spoken language.
  
We simulate grounded language learning in a naturalistic setting, where 
connection between the linguistic and visual modality is not always strong 
and is potentially confounded by correlations with non-semantic aspects of 
the speech signal. Our experimental results suggest that despite the 
challenges inherent to the naturalistic aspects of our training dataset, a 
simple bimodal architecture can capture aspects of visual meaning of individual 
words as well as full utterances, and generalize well to narrative utterances
featuring a single unseen speaker and a descriptive rather than
conversational style. Our analyses show that generalization is substantially
boosted by fine-tuning audio representations pretrained on unlabeled
single-modality speech data. Fine-tuning a pretrained video encoder
also makes a contribution, but is less crucial to generalization from
dialog to narration.
%
We also investigate the role of temporal information in learning form-meaning 
mappings. Our various experimental conditions show that having access to 
temporal information facilitates learning, except for very short video segments. 

To better understand what aspects of language learning are affected by 
the dynamic nature of video data, we need to to carry out in-depth analyses of learned 
representations on sub-word, lexical, and phrasal levels. It would also be 
worthwhile to figure out the details of how specifically temporal information 
in video contributes to acquiring linguistic knowledge.  Some analyses in 
this direction are currently constrained by the size of the evaluation 
dataset, and more large-scale datasets are needed in the future.

We model the acquisition of spoken language from 
language-internal correlations as well as from grounding in vision 
by fine-tuning an audio encoder pretrained on read speech. This 
approach is rather simplistic and does not match the real experience of 
language learners. It would be interesting to make the setting
more realistic by using pretraining data which reflect a young
learner's experience more closely, and to realistically interleave learning via
self-supervision from speech and via grounding in vision.
Ideally we would also want to dispense with supervised pretraining of 
the video encoder and rather use a model pretrained in a
self-supervised way also for this modality.

